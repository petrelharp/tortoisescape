\documentclass{article}
\usepackage{amsmath,amssymb}
\newcommand{\R}{\mathbb{R}}
\newcommand{\E}{\mathbb{E}}
\renewcommand{\P}{\mathbb{P}}
\newcommand{\bone}{\mathbf{1}}

\begin{document}

Suppose that 
\begin{align}
  G = \sum_k \alpha_k A_k,
\end{align}
where $\alpha \in \R^n$
and each $A_k$ is a generator matrix,
i.e.\ $(A_k)_{xy} > 0$ for $x\neq y$ and $A_k \bone = 0$ for each $k$.

Consider a random walk driven by $G$, and let
\begin{align}
  H_{xy} = \E^x[\tau_y] = \text{ ( mean hitting time of $y$ from $x$ ) } .
\end{align}
Then  $H_{xy} \ge 0$ and
\begin{align}
  \sum_x G_{zx} H_{xy} &= -1 \quad \text{ for each } z \neq y \\
        H_{yy} = 0 .
\end{align}

Now suppose we are given the $A_k$ and $H_{xy_i}$ for all $x$ and some set of $\{y_i\}_{i=1}^m$.
We could find $\alpha$ to minimize
\begin{align}
  F(\alpha) = \sum_{i=1}^m \sum_{z \neq y_i} \left( \sum_k \alpha_k \sum_x (A_k)_{zx} H_{xy_i} + 1 \right)^2 .
\end{align}
Differentiating this with respect to $\alpha_j$,
we get
\begin{align}
  \partial_{\alpha_j} F(\alpha) = 2 \sum_{i=1}^m \sum_{z \neq y_i} \left( \sum_x (A_j)_{zx} H_{xy_i} \right) \left( \sum_k \alpha_k \sum_x (A_k)_{zx} H_{xy_i} + 1 \right) .
\end{align}
For each $i$ and $j$ let $B^{ij}$ be the vector
\begin{align}
  B^{ij}_z = \sum_x (A_j)_{zx} H_{xy_i} ,
\end{align}
and let $B^{ij} \cdot B^{ik} = \sum_{z \neq y_i} B^{ij}_z B^{ik}_z$, etcetera.
The derivative above is then
\begin{align}
  \partial_{\alpha_j} F(\alpha) = 2 \sum_{i=1}^m \left( \sum_k \alpha_k B^{ij} \cdot B^{ik} + B^{ij} \cdot \bone \right) .
\end{align}
Setting these to zero for each $j$ results in the system of linear equations
\begin{align}
  Q \alpha = b
\end{align}
where
\begin{align}
    Q_{jk} &= \sum_{i=1}^m  B^{ij} \cdot B^{ik} \\
    b_j &= - \sum_{i=1}^m  B^{ij} \cdot \bone .
\end{align}

For computation, we could compute the matrix
\begin{align}
  B^j = A_j H ,
\end{align}
indexed by $z$ and $i$,
and then set
\begin{align}
  B^j_{y_i i} := 0 \quad \text{for each } i ,
\end{align}
and then $C^j = B^j \bone$, i.e.
\begin{align}
  C^j_z = \sum_{i=1}^m B^j_{z i} .
\end{align}
Then we would have
\begin{align}
    Q_{jk} &= C_j \cdot C_k \\
    b_j &= C_j \cdot \bone' .
\end{align}


\end{document}
